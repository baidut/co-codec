\documentclass[11pt]{article}
\usepackage[czech]{babel}
\usepackage[utf8]{inputenc}
\usepackage{a4wide}

\usepackage[pdf]{pstricks}

\title{Aritmetické kódování}
\author{Tomáš Maršálek}
\date{\today}

\begin{document}
\maketitle

\section{Zadání}
Naprogramujte v ANSI C přenositelnou konzolovou aplikaci, která jako vstup
načte z parametru na příkazové řádce jmého souboru, který se má
zkomprimovat/dekomprimovat, a jméno výstupního souboru, do kterého uloží
zkomprimovaná/dekomprimovaná data. Program bude data komprimovat technikou
aritmetické komprese (angl. Arithmetic Coding). Zda se bude provádět komprese
nebo dekomprese se určí přepínačem na příkazové řádce.

\section{Analýza úlohy}
Aritmetické kódování patří do kategorie takzvaných entropických kodérů. Ty ke
kompresi využívají pravděpodobnostní rozdělení znaků řetězce. Příkladem je
Morseova abeceda nebo Huffmanovo kódování. Obě tyto techniky přiřadí
frekventovanějším znakům kratší kód a naopak delší méně frekventovaným.  Tato
kategorie kodérů je již vyřešený problém, protože Huffmanovo kódování přiřadí
každému znaku optimální kód pro daný vstup. Teoreticé maximum zakódování
popisuje Shannonova věta o kódování bez šumu, podle ní je optimální délka
$\log_2 \frac{1}{P}$ bitů na znak. $P$ je pravěpodobnost výskytu znaku.  Huffman,
přestože určuje optimální kódy, teoretického minima dosáhne pouze pokud je
pravděpodobnost znaku mocnina dvou, logaritmus pak vyjde celočíselně.
Aritmeticé kódování však funguje na jiném pricipu a díky tomu je možné znakům
přiřadit i kódy neceločíselné délky.  Zakóduje celou zprávu jako celek a tedy
ve výsledku mohou mít jednotlivé znaky i zlomkové délky.

\subsection{Princip}
Vstup je mapován na číslo v intervalu $[0, 1)$. Interval je rozdělen na úseky
odpovídající šířkou pravděpodobnostem znaků. Vstupní znak poté určí
odpovídající subinterval, čímž o sobě zaznamená jednoznačnou informaci. Tento
subinterval je rozdělen stejným způsobem podle pravděpodobností a další znak
vybere svůj subinterval. Tímto způsobem se zakódují všechny znaky. Aritmetické
kódování vlastně využívá nekonečné dělitelnosti množiny reálných čísel.

\paragraph{Příklad}
Pravděpodobnosti výskytu znaků jsou $A : 0.3$, $B : 0.5$, $C : 0.2$. Kódované
slovo je $BAC$. V každém kroku je zvýrazněn právě vybraný interval a jeho
rozdělení. Poté co určíme poslední subinterval, vybereme z něj libovolné číslo.
To bude tvořit aritmetický kód zprávy.
\mbox{} \\

\psset{unit=1mm}
\begin{pspicture}(150, 30)
	% first line
	\thicklines
	\put(0,   30){\line(1, 0){150}}   % main line
	\put(0,   28){\line(0, 1){4}}     % label 0
	\put(150, 28){\line(0, 1){4}}     % label 1
	\thinlines
	\put(-1,  24){$0$}
	\put(149, 24){$1$}

	\put(45,  29){\line(0, 1){2}}
	\put(120, 29){\line(0, 1){2}}

	\put(22,  33){$A$}
	\put(82,  33){$B$}
	\put(135, 33){$C$}

	% dotted lines
	\psline[linestyle=dotted](0, 30)(45, 15)
	\psline[linestyle=dotted](150, 30)(120, 15)


	% second line
	\put(0,    15){\line(1, 0){150}}   % main line
	\put(0,   13){\line(0, 1){4}}     % label 0
	\put(150, 13){\line(0, 1){4}}     % label 1
	\put(-1,  9){$0$}
	\put(149, 9){$1$}

	\put(67, 14){\line(0, 1){2}}
	\put(105,14){\line(0, 1){2}}
	\thicklines
	\put(45, 15){\line(1, 0){75}}
	\put(45, 14){\line(0, 1){2}}
	\put(120,14){\line(0, 1){2}}
	\thinlines

	\put(53,  18){\small $BA$}
	\put(83,  18){\small $BB$}
	\put(110, 18){\small $BC$}

	% dotted lines
	\psline[linestyle=dotted](45, 15)(63, 0)
	\psline[linestyle=dotted](120, 15)(67, 0)

	% third line
	\put(0,    0){\line(1, 0){150}}   % main line
	\put(0,   -2){\line(0, 1){4}}     % label 0
	\put(150, -2){\line(0, 1){4}}     % label 1
	\put(-1,   -6){$0$}
	\put(149,  -6){$1$}

	\put(45, -1){\line(0, 1){2}}
	\put(120,-1){\line(0, 1){2}}

	\put(67, -1){\line(0, 1){2}}
	\put(105,-1){\line(0, 1){2}}

	\put(51,-1){\line(0, 1){2}}
	\thicklines
	\put(63, 0){\line(1, 0){4}}
	\put(63,-1){\line(0, 1){2}}
	\put(67, -1){\line(0, 1){2}}
	\thinlines

	\put(45,  2){\tiny $BAA$}
	\put(52,  2){\tiny $BAB$}
	\put(62,  2){\tiny $BAC$}

	% encoded number line
	\thicklines
	\put(64.5, -5){\line(0, 1){6}}
	\put(64.5, 5){\line(0, 1){28}}
	\thinlines

	\put(61, -9){$0.43$}
\end{pspicture}
 \\[1cm]

Libovolné číslo z daného intrevalu vybíráme s co nejkratší délkou, v tomto případě
je kód zprávy $0.43$.

\paragraph{}
Pro kratší kódy chceme, aby se vybíraly co nejdelší subintervaly. Pak zůstane 
délka intervalu velká a nebudeme potřebovat tolik číslic k zapsání výsledného čísla.
Proto je důležité mít správně zmapované pravděpodobnosti znaků, více frekventované
znaky získají širší interval a tím pádem dochází ke kompresi. Kdyby měl program
k~dispozici informaci, že všechny znaky jsou stejně pravděpodobné, nedošlo by k žádné
změně velikosti po zakódování (otestováno).

\section{Implementace}
\section{Závěr}
\end{document}
